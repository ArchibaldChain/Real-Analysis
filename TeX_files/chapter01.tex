\chapter{Cardinality}
\newtheorem*{myDef}{Definition}
\newtheorem*{myTheo}{Theorem}
\newtheorem*{myPf}{Proof}
\newtheorem*{myLemma}{Lemma}
\newtheorem*{myProp}{Proposal}

\section{Cardinal Number}
\begin{myDef}[Cardinality]
Let $A$, $B$ be two (non-empty) sets, we say  $\abs{A}\leq\abs{B}$ if there exists a injective map $f:\ A\rightarrow B$
\\
Similarly we have $\abs{A}\leq \abs{B} \Leftrightarrow$ there exists a surjective map $g: B\rightarrow A$
\\
We say $\abs{A} = \abs{B} $
\end{myDef}


\begin{myLemma}[Set Decomposition Under Mapping]
	Let $f: X\rightarrow Y,\ g: Y\rightarrow X$ there exists decomposition such that,
	\[X = A\cup A^{\sim}, Y=B\cup B^{\sim}\]
	where $f(A)=B, g(B^\sim)=A^\sim, A\cap A^\sim=\emptyset$ and $B\cap B^\sim=\emptyset$
\end{myLemma}
\begin{myPf}
	For a subset $E$ of $X$, W.O.L.G. $Y\backslash f(E)\neq\emptyset$, if $E$ satisfy,
	\[E\cap g(Y\backslash f(E))=\emptyset\]
	we call $E$ a seperate set of $X$, now denote the set of all seperate set as $\Upgamma$, and make union,
	\[A=\bigcup_{E\in \Upgamma}E\]
	We have $A\in \Upgamma$, Actually for any $E\in \Upgamma$, as $A\supset E$, so from,
	\[E\cap g(Y\backslash f(E))=\emptyset\]
	we know that $E\cap g(Y\backslash f(A))=\emptyset$, since $A$ is larger than $E$, so actually $g(Y\backslash f(A))\subseteq g(Y\backslash f(E))$, thus $A\cap g(Y\backslash f(A))=\emptyset$. This shows that $A$ is also a seperate set in $X$ and the largest element in $\Upgamma$.
	\\
	Now let $f(A)=B, Y\backslash B=B^\sim$ and $g(B^\sim)=A^\sim$. First we know that.
	\[Y=B\cup B^\sim\]
	Secondly, as $A\cap A^\sim=\emptyset\Leftrightarrow A\cap g(Y\backslash f(A))=\emptyset$, so we know $A\cup A^\sim=X$.
	\\
	We can assume $A\cup A^\sim \neq X$, then there exists $x_0\in X$ such that $x_0\notin A\cup A^\sim$. Let $A_0=A\cup {x_0}$ we have,
	\[B=f(A)\subset f(A_0), B^\sim \supset Y\backslash f(A_0)\]
	so that $A^\sim \supset g(Y\backslash f(A_0))$, which means $A$ and $g(Y\backslash f(A_0))$ do not intersect. So
	\[A_0\cap g(Y\backslash f(A_0))=\emptyset\]
	\\($A_0$多了一个元素$x_0$,但该元素在$A^\sim$ 中不存在并且$A^\sim \supset A_0^\sim=g(B_0^\sim) = g(Y\backslash f(A_0))$)
	which is contradict to $A$ is the largest element in $\Upgamma$
\end{myPf}
\begin{myTheo}[Schröder–Bernstein theorem]
	If $\abs{X}\leq\abs{Y}$ and $\abs{Y}\leq\abs{X}$, then $\abs{X}=\abs{Y}$
\end{myTheo}
\begin{myPf}
	We need to show if there exists an injective map $f: X\rightarrow Y$ and an injective map $g: Y\rightarrow X$ then there exists a bijective map $h: X\rightarrow Y$.
	\newline
	\newline
	Define $X=A\cup A^\sim, Y=B\cup B^\sim, f(A)=B(Surjective\ to\ B), g(B^\sim)=A^\sim(Surjective\ to\ A^\sim)$(Using decomposition lemma)
	\newline
	For any $a\in X$, define a map $h$.
	\[
	h(x)=\left.
	\begin{cases}
	f(x) & x\in A\\
	g^{-1}(x) & x\in A^\sim
	\end{cases}
	\right.
	\]
	which shows $X\sim Y$
\end{myPf}
\begin{myDef}[Arithmetic Operation of Cardinal Number]
	Suppose $a$, $b$ are two cardinal numbers, where $a=\abs{A}$, $b=\abs{B}$.
	\begin{enumerate}
		\item $a+b\triangleq\abs{A\cup B}$ where $A$, $B$ are disjoint sets.
		\item $a\cdot b\triangleq \abs{A\times B}$
		\item $a^b=\abs{A^B}=\prod_{B}A$ where $A^B=\{\text{all maps }\phi: B\rightarrow A\}$
	\end{enumerate}
\end{myDef}
\begin{myProp}
	$c=m^{\aleph_0}, \forall m\in \mathbb{N}, m\geq2$
\end{myProp}
\begin{myPf}
	We view R.H.S. as $\abs{\{0,1,2,...,m-1\}^{\mathbb{N}}}=\abs{\{f:\mathbb{N}\rightarrow\{0,1,...,m-1\}\}}$. Actually, R.H.S. can be viewed as a map from $i_{th}$ digit index to the $i_{th}$ digit itself and L.H.S. as $\abs{(0,1]}$.
	\\
	Recall that $\forall r \in (0,1]$ we have a sequence $\{r_n\}$, each $r_n\in\{0,1,...,m-1\}$ such that
	\[r=\sum_{n=1}^{\infty}\frac{r_n}{m_n}\]
	(闭区间套: the principle of nested intervals)
	\\
	The sequence $\{r_n\}$ is unique if we require it has infinite many non-zero numbers.
	\\
	This means that we have an injective map.
	\[\Upphi: (0,1] \rightarrow \{0,1,...,m-1\}^\mathbb{N}\]
	\[Im\Upphi=\{\text{f: there are infinitely many $n$ with $f(n)\neq0$}\}\]
	Let $A_N=\{f:\exists\ N\ s.t.\ f(n)=0\ \forall n>N\}$, so $\abs{A_N}=m^N<\infty$
	\[\abs{(Im\Upphi)^\complement}=\abs{\bigcup_{N=0}^\infty A_N}=\aleph_0\]
	So, now we have shown,
	\[L.H.S.=c=c+\aleph_0=m^\aleph_0=R.H.S.\]
\end{myPf}
\begin{myDef}[Power Set]
	$P(A)\triangleq \{subsets\ of\ A\}\triangleq\{0,1\}^A=\{f: A\rightarrow \{0,1\}\}$
\end{myDef}
\begin{myTheo}[$P(A)>\abs{A}$]
	First, $P(A)\geq\abs{A}$, because there exists a injective map $f: a\rightarrow \{a\}$. we will show $\abs{P(A)}\neq \abs{A}$, hence $\abs{P(A)}>\abs{A}$.
	\\
	Otherwise, $\abs{P(A)}=\abs{A}$ and hence there exists a bijective map $\phi: A\rightarrow P(A)$.
	\\
	Consider the subset $B=\{a\in A|a\notin \phi(a)\}$, thus $B\in P(A)$.
	\\
	Let $b$ be the pre-image of $B$ under $\phi$.
	\\
	If $b\in B$ then by the construction of $B$, then $b\notin \phi(b)=B$.
	\\
	If $b\notin B$ then by the construction of $B$, then $b\in \phi(b)=B$.
	\\
	Therefore, such bijective map $\phi$ does not exist.
\end{myTheo}
\section{Construct Real Number}
With Cauchy sequence, we can define real number (almost) as a sequence of rational number approximating a `real number'. But we still need to deal with different Cauchy sequences converging to the same number. Hence we will need to group Cauchy sequences into sets, all of which have the same `tail behavior', and we will define a real number to be such a \emph{a set of Cauchy sequences}.
\begin{myDef}[Equivalence Relation]
	Let S be a set of objects. A relation among pairs of elements of S is said to be an \textbf{equivalence relation} if the following three properties hold:
	\begin{enumerate}
		\item Reflexivity: $\forall s\in S$, $s$ is related to $s$.
		\item Symmetry: $\forall s, t \in S$, if $s$ is related to $t$, then $t$ is related to $s$.
		\item Transitivity: $\forall s,t,r \in S$, if $s$ is related to $t$ and $t$ is related to $r$, then $s$ is related to $r$.
	\end{enumerate}
\end{myDef}
\begin{myTheo}
	Let $S$ be a set, with an equivalence relation on pairs of elements. For $s\in S$, denote by $[s]$ the set of all elements in $S$ that are related to $s$. Then for any $s,t\in S$, either $[s]=[t]$ or $[s]$ and $[t]$ are disjoint.(Proof is skipped, quite trivial, just suppose there exists $r\in [s]$ and $r\in [t]$)
\end{myTheo}
\begin{myDef}[Real Number]
	Real numbers are constructed as \textbf{equivalence classes of Cauchy sequence}.
\end{myDef}
Let $\mathscr{C}_\mathbb{Q}$ denote the set of all Cauchy sequences of rational numbers. We must define an equivalence relation on $\mathscr{C}_\mathbb{Q}$.
\begin{myDef}
	Let $(a_n)$ and $(b_n)$ be in $\mathscr{C}_\mathbb{Q}$. Say they are equivalent (i.e related) if $a_n-b_n\rightarrow 0$. i.e. if the sequence $(a_n-b_n)$ tends to $0$.
\end{myDef}